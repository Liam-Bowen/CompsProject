\documentclass[10pt,twocolumn]{article} 


\title{Literature Review}
\author{Liam Bowen}
%\date{}							% Activate to display a given date or no date

\begin{document}
\title{Literature Review}
\author{Liam Bowen}

\maketitle

\section{Problem Context}
	
When thinking about my plan for comps, I was dancing around several ideas. The one thing I knew was that I wanted to involve baseball in the project somehow. A couple of my friends and I were discussing the upcoming baseball season and how the current lockout was going to cancel games and make the season shorter. We started talking about how the lockout was going to make the fantasy baseball season shorter as well, and the general consensus was that fantasy baseball was going to be more fun because there would not be as many weeks for us to labor through. That’s when I came up with the idea to make fantasy baseball more fun. 	
		
Fantasy baseball is a game where the players are owners of a virtual baseball team. The players compete during the season to have the best team. Each week is a game between players as they compete for the best record. Results and scoring are based off of stats from real life. The last few years of playing fantasy baseball with my friends had become very repetitive. I was noticing a trend that every year after the first five or so weeks, people stopped running their teams. The season would progress more and the only people continuing to play were the ones who had a chance to win it all. While there have been solutions outside the game to prevent players from quitting such as punishments for the player who finishes last, my friends and I have never followed through with instilling a punishment on the biggest loser. 

So, I wanted to come up with a way to keep players involved in a fantasy baseball season as long as possible. To brainstorm ideas, I turned to fantasy football. Year after year, fantasy football is a game my friends and I are always extremely excited to play. A couple of years ago, we started doing what is known as a keeper league. The keeper league is a league where each owner gets to pick a handful of players from their roster to keep for the next year. This new way of playing was fun and always got us excited and thinking about the next year. The football season is about ten weeks shorter than a full baseball season, so fantasy players can keep engaged in the game longer as the season is not as drawn out. 

I thought to myself, how could I implement an idea like this to make fantasy baseball a game that does not get left in the dust. So, my main goal will be to implement a midseason redraft of teams. I am going to use the idea of a keeper league where an owner gets to keep certain players from their team, but at the midway point of the season. I hope the redraft can provide a possible second life for owners whose teams are struggling. The redraft allows for them to select new players to their team that were previously on a more successful team’s roster. I believe that struggling owners will stay more engaged leading up to the second draft because they know that if they can stay as close to the teams in front of them in standings, they might be able to improve their team and then pass them in the second half.

I thought that this idea might not be enough to keep people playing throughout the entire year however. The redraft might succeed in only keeping engagement through the midpoint and a few more weeks if the lower ranked players still struggle. So I had to think of another possible fix to add to this. I thought, what if there was an incentives system in place for when teams ranked lower beat teams better than them. What I’ve come up with is a steal a player feature. The steal a player feature means that if a team with a worse record beats a team with a better record than them, they get to pick a player from the other team and get to take that player. In return, they have to send a player back from the same position. I really like this idea because a lower ranked team can build hope later in the season. Beating a team better than them allows for them to not only improve their team, but it also makes the opposing team a little bit worse. The playing field begins to even out, and the lower ranked teams will start to climb standings and keep close in the standings all the way until the end of the year. 

Overall, I believe that these two strategies can successfully complete my goal. Players will continue to update their teams and compete with their friends to try and win the league throughout the entirety of the season. 

\section{Technical Background}

I do not have much technical background in either web app design or game design. My experience is limited in designing a web app, so the visual appeal will be something I need to spend some time on to make sure that the site is cohesive and easy to navigate. To help me with this, I am going to reference not only ESPN’s fantasy baseball website, but also Yahoo Sports fantasy baseball website as well. Both of these sites succeed in their visual appeal. The sites have clear and concise links to pages and users can follow information easily through drop downs and linking pages. 

Part of my work will include downloading statistical data of players throughout the season. I will need to not only download the statistics off of a website such as ESPN, but I will need to then implement the stats in a similar way that users can easily sort by certain metrics and compare players to one another. 

I want the layout of my fantasy baseball page to be similar to that of ESPN’s, meaning I am going to need to create a fluid way for users to switch between pages that show the league standings, their team roster, and available players to add. 

My experience in this sector is fairly limited. I have a small amount of experience working with html to design a website, but my knowledge of web design stops there. I have experience with statistics, and I have used excel to collect and analyze data. I have done statistical coding through Python where I’ve uploaded .csv files and run programs to study trend-lines. I won’t need to dive this deep with the statistical side of the project, but I do feel that my prior experience with excel will be able to help.

\section{Prior Work}

Prior work that I have seen similar to what I want to do includes the previously mentioned ESPN (https://www.espn.com/fantasy/baseball/). ESPN does a great job with fantasy sports and has been at the forefront of fantasy sports for years. Yahoo Sports(https://baseball.fantasysports.yahoo.com/)  has been around for many years, and it still continues to be used for fantasy baseball leagues to this day. Both sites have tried keeping players more interested in fantasy baseball by giving the league commissioner the ability to alter more about the league. They can change how many points each statistic is worth. They can add or drop statistics to make room for more advanced data that has become popular the last few years. However, their versions of fantasy baseball have been relatively the same game since launch. 

Another site that I have studied to help in my work is dynastyleaguefootball.com. While this is not a fantasy baseball site, it is a website that does a great job of making the game more fun. The dynasty league lasts year round as owners keep their team year to year. They argue that this allows for a greater sense of team ownership which makes owners consistently more involved. This is a newer website, but it has become popular as more and more people feel that retaining certain players on their team each year makes the game more entertaining and brings forth a whole new way of strategizing rosters. 

\end{document}  
